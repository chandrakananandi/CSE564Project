\documentclass{article}
\usepackage[english]{babel}
\usepackage{fullpage}
\usepackage{graphicx}
\usepackage{enumitem}
\usepackage{url}
\usepackage[usenames, dvipsnames]{color}
\usepackage[utf8]{inputenc}
\title{\textbf{CSE 564 Project Report} 
\\
\begin{center}
\LARGE{Design and Verification of Home-automation Systems}
\end{center}\vspace{-2ex}}
\author{Jeanette Daum (jdaum), Chandrakana Nandi (cnandi) }
\date{March 14, 2016}

\begin{document}

\maketitle
\begin{abstract}
One of the applications of Internet of Things which is gaining popularity is Home Automation. These systems are typically made up of devices which can be controlled remotely by the home inhabitants or can communicate with each other and act autonomously. An objective for smart homes is ensuring their security and correctness because violation of these properties can directly affect the lives of the inhabitants. What if the controller of one device incorrectly sends commands to another device or sends the command to a device under incorrect conditions? In this project, we design and implement an abstract architecture for modelling smart homes in a way that makes it easier to reason about their security and correctness, define different types of policies and specify them for three different smart systems and develop a static analyzer that verifies whether the implementation of the smart systems complies with the specified policies. 

\end{abstract}
\section{Introduction}
Home automation systems are a new trend that is just starting to gain momentum and will soon have a huge impact on our everyday lives~\cite{samsung, homekit, echo, wink, homeos}. Typically, these systems consist of devices interacting with each other. These devices can take actions autonomously or they can be controlled remotely by the home inhabitants. An example is a garage door that opens automatically whenever the car is within a certain distance, but also opens and closes when the user gives a remote command by pressing a button. 

For our project, we consider a smart home to be a collection of interacting devices acting autonomously without requiring the user to intervene. Nowadays most of the existing systems require user interaction and remote control but we expect future systems to become more autonomous to further assist its inhabitants~\cite{homekit}.

A challenge for smart homes is to ensure that there are no security loopholes because they can be exploited by malicious attackers to take control of the house and harm the inhabitants~\cite{yoshi, jung}. Before smart homes become as widespread as smart phones are today, it is necessary for the home automation industry to address these concerns. For autonomous systems, security and correctness is even more crucial because the user relies on the system and may not have much control over it. 

In this project, our goal is to ensure that a system of interacting smart devices satisfies a set of security properties. We are concerned with security at the device level and not with the security of the communication protocols or the underlying operating system. We aim to solve the problem of security for smart homes from the point of view of the manufacturers of the smart devices so that they can guarantees that their devices are secure and correct. Section~\ref{subsec:roles} describes the roles played by manufacturers of devices, users and major organizations (such as Samsung, Apple, etc.) in our model in detail.

We have modelled a smart home as a collection of three types of devices: controllers, sensors and dumb-devices. The controllers send commands to the dumb-devices based on values of sensors which measure different parameters such as temperature, state of a dumb-devices etc. The dumb-devices simply execute the commands sent to them.
We defined a set of security policies and specified them for three different smart systems: an automatic laundry system, an automatic garage door system and a thermostat. We developed a static analyzer that verifies that the three smart systems adhere to their respective policies. By conducting a preliminary experiment explained in section~\ref{sec:analysis}, we show that for smart homes, dynamically enforcing security policies can lead to a large overhead which may affect the reliability of the system. This motivates our decision of doing static program verification to provide compile time guarantees that a smart home is secure. Our main contributions are:
\begin{itemize}[topsep=0pt,itemsep=0ex,partopsep=1ex,parsep=1ex]
    \item Designing a smart home architecture, implementing a prototype of the architecture and three smart systems based on it.
    \item Assigning roles to different organizations involved in the smart home business. 
    \item Defining and formalizing different types of security and correctness policies and specifying them for the three smart systems.
    \item Developing a static analysis tool for verifying the code of the smart systems.
\end{itemize}
  
\section{Related work}
Some researchers have suggested general solutions for ensuring security for smart homes such as choosing a stronger password for protecting the devices or modifying their security settings according to one's needs~\cite{semantec}. Such solutions rely on the user's awareness about security vulnerabilities and do not address the problems specific to smart homes such as devices with different functionalities interacting without user intervention. Hence they cannot guarantee that a smart home will always behave correctly. Another proposed solution~\cite{al2000secure} is based on dynamic enforcement. The problem with dynamic solutions is that if a security breach is encountered during execution, the service has to be stopped. This is not convenient from the user's point of view because it affects the reliability of the system. Further, as we show in this paper, enforcing security and correctness policies dynamically leads to an execution time overhead which is not desirable in home automation. Consider a situation where there is a fire in the house and the inhabitants want to evacuate as soon as possible. If there is dynamic policy checking, that might cause delay in the opening of the front door which can affect the physical safety of the people. 
In this paper, we specify security and correctness policies for the smart devices in a home and verify them statically thereby providing compile time guarantees that the entire smart home is secure and correct.

\section{Architecture}
\label{sec:archi}
We present a high-level modular architecture for designing smart homes which classifies devices into three categories depending on their functionality: \textit{sensors}, \textit{controllers} and \textit{dumb devices}. There are two types of messages exchanged among the devices: 
\begin{itemize}[topsep=0pt,itemsep=-1ex,partopsep=1ex,parsep=1ex]
    \item \textit{informational}: Sensors broadcast information such as the state of a dumb device or the temperature in the house.  
    \item \textit{commands}: Controllers send commands to the dumb devices telling them what to do.
\end{itemize}

\begin{figure}
\begin{center}
\includegraphics[scale=0.4, trim= 0 600 0 0]{figure1.pdf}
\caption{\color{RoyalBlue}{Sensors}, \color{Maroon}{controllers,} \color{black}{and} \color{LimeGreen}{dumb devices}. %\color{black}{The thermostat sends commands depending on values from two sensors: the thermometer and the location sensor. The door controller also sends commands depending on the values of two sensors: the location sensor and the current state of the door.}
}
\label{fig:arch}
\end{center}
\end{figure}

\subsection{Sensors, controllers and dumb devices}
\label{subsec:devices}
\textit{Sensors} measure properties like temperature, motion, state of a dumb device, etc. and notify controllers depending on them. Sensors can be inside or outside the house. Examples of sensors inside the house are: motion detectors, the thermostat etc. and outside the house are: a camera outside the main door or the GPS sensor in the owner's smart phone when the owner is outside. 

\textit{Controllers} send commands directly to some of the dumb devices depending on the values of one or more sensors. 
%They have a set of commands they can send and a set of \textit{dependency policies} which specify the conditions (sensor values) under which they can send a command to a dumb device. Both of these are the specifications against which a controller's implementation is verified. 

%A dumb device %belongs to some category such as room-heater, air-conditioner, light-bulb, garage-door etc. and 
%has a set of actions it can execute, such as on, off, open, close etc. 
\textit{Dumb devices} do not send messages to other dumb devices, sensors, or controllers. They atomically execute the commands sent to them by any controller. 

It is not necessary for the sensors, controllers, and dumb devices to be physically separate from each other: 
a controller may be fit to a dumb device or a dumb device may have sensors attached to it. For formal reasoning about the system, we treat them as separate entities. Figure~\ref{fig:arch} illustrates the architecture of a simple example smart system. 

\subsection{Role of major organizations, users and device manufacturers}
\label{subsec:roles}
We consider the smart home market to comprise of three different groups playing different roles:
\begin{itemize}[topsep=0pt,itemsep=0ex,partopsep=1ex,parsep=1ex]
\item \textit{Major organizations}: We define major organizations as companies such as Samsung~\cite{samsung}, Apple~\cite{homekit}, etc. who provide platforms (hubs or apps) for integrating different smart devices and communication protocols to make them interact with each other. 
\item \textit{End users}: They are the home inhabitants who own (or want to own) a smart home. An end user might want to buy a hub (or app) from one of the existing \textit{major organizations} to plug in different smart devices and convert their ordinary home to a smart home.

\item \textit{Device manufacturers}: They are the companies such as Philips, Withings, etc. who develop \textit{controllers}, \textit{dumb-devices} and \textit{sensors}~\cite{workswithsmartthings, workswithhomekit} which can be integrated to the platforms developed by the \textit{major organizations}. The major organizations themselves can also be device manufacturers; both Samsung and Apple manufacture their own smart devices.
\end{itemize}

In our smart home model, we consider the major organizations to 

%Verification of the policies we defined ensures that (a) a controller does not send commands to a dumb device it owns under the wrong conditions, (b) it only send correct commands to the correct dumb devices, and (c) adding a new device to the house does not violate any security or correctness policy. \\

%\subsection{Device discovery}
%\label{subsec:discovery}
%A controller can discover other controllers, sensors, and dumb devices and the commands the latter can execute. If these commands match with the commands the controller can send, then it can take charge of the dumb device. Every controller stores a table mapping the IDs of the controllers to a list of pairs of dumb device IDs and commands the controllers may send to them. Two example entries are:\\
%\{\texttt{thermostatID:\{(roomheaterID, start\_roomheater), \\ 
%\hspace*{2.5cm}(roomheaterID, stop\_roomheater),\\ 
%\hspace*{2.5cm}(airconditionerID, start\_airconditioner), \\
%\hspace*{2.5cm}(airconditionerID, stop\_airconditioner)\}\\
%\hspace*{0.2cm}doorcontrollerID:\{(doorID, open\_door),\\
%\hspace*{3.0cm} (doorID, close\_door)\}\}
%}\\

%\noindent The controllers stay in sync with identical contents in their tables. If there is no controller that can send the commands a dumb device can execute, no message will be sent to it unless a suitable controller for it is installed. %When this new controller is installed, it discovers all the other dumb devices, sensors and controllers in the house (and is likewise discovered by all the existing controllers). It updates its table with the entries from all the other controllers and adds the dumb device with the corresponding commands against its ID in the table, after which the tables of all the other controllers are updated. 
%Once a dumb device and a command it can execute are chosen by a controller, no other controller may send the same command to it. This is verified by the control policy described in \S\ref{subsec:control}.

\section{Threat Model}
Our threat model is an attacker trying to compromise the controllers to make them send commands to dumb-devices under incorrect conditions. This is a security vulnerability because an attacker can change the sensors and/or their values to send commands to dumb-devices under wrong conditions and make the smart home behave in unexpected ways. For example, consider a garage door which is only supposed to open when the owner' car is within 50m from it. An attacker can 
%\begin{itemize}
    %\item 
    %\item send ``illegal" commands to the dumb devices. An illegal command could be a wrong command to a dumb device that the controller controls, a right command to a dumb device that the controller does not control or a wrong command to a dumb device that the controller does not control.
%\end{itemize}
In other words, we do not trust the controllers and hence, verify their code statically to ensure that their behavior cannot be exploited. For the verification, we check the source code against policies specified in separate files. We assume that the specifications are trusted.\\

\noindent To scope our research, we assume that the sensors and dumb-devices are not compromised and that the sensors always store the latest values of the properties they measure. We also assume that the dumb devices trust the commands they receive from the controllers and atomically execute them. Additionally, we consider the communication protocol to be trusted and do not focus on man-in-the-middle attacks. 
While these are strong assumptions, we will show that even under them, we can still find a lot of serious vulnerabilities in a smart home. 


% policies are safe/cannot be compromised
% focus on attacks that compromise controllers by malicious parties -> try to control the dumb device illegally through the controller
% sensors are not compromised, always updating % broadcasting the latest sensor values
% dumb devices are trusted, they also trust the commands they get from the controllers and execute them automatically 


\section{Security Policies}
We have specified three types of security policies that should be satisfied by a smart home automation system. Our policies focus on the behaviour of the controllers and their interaction with the other devices in a smart home. Following are the different types of policies a smart home must satisfy: 
\begin{itemize}
\item Dependency policy (\S\ref{subsec:dep}): We formally specify policies for each controller that state under what circumstances (sensor values) a controller is allowed to send a command.

\item Control policy (\S\ref{subsec:control}): A controller should only send correct commands to correct dumb devices. For example, the bedroom light controller should not control the lights in the living room.

\item New device policy (\S\ref{subsec:newitem}): This policy states that when a new device is added to a house, verification of the above two policies is only needed if the new device is a controller and only verifying the new controller ensures that the entire house is secure and correct, as long as all the existing controllers are already verified.
\end{itemize}

All three policies are important when it comes to ensuring the security of a smart home. However, in the interest of time, we are focusing our verification on just one policy. Our main focus is on the \textit{Dependency policy} because many security violations occur when this policy is violated. 

\begin{table*}[t]
\begin{center}
\begin{tabular}{|p{4cm}|p{6cm}|p{4.5cm}|}
\hline
controller & commands & sensors \\ \hline
\tiny{GARAGE\_DOOR\_CONTROLLER} & 
\small{open\_garagedoor, close\_garagedoor}\tiny &  
\tiny{IS\_GARAGE\_OPEN, \newline IS\_CAR\_INSIDE\_GARAGE, \newline IS\_CAR\_RUNNING, \newline IS\_OWNER\_INSIDE\_CAR,\newline CAR\_DISTANCE (double), \newline CAR\_SPEED (double)} \\ \hline

\tiny{LAUNDRY\_MACHINE\_CONTROLLER} &
\small start\_washer, \small stop\_washer & 
\tiny{IS\_WASHER\_ON}, \newline \tiny{IS\_EMPTY, \newline IS\_DOOR\_CLOSED, \newline IS\_CLEANED} \\ \hline

\tiny{THERMOSTAT} & 
\small start\_roomheater,\newline \small stop\_roomheater,\newline\small start\_airconditioner, \newline \small stop\_airconditioner & 
\tiny{IS\_ROOMHEATER\_ON}, \tiny{IS\_AIRCONDITIONER\_ON},
\tiny{IS\_HOUSE\_EMPTY, \newline IS\_WINDOW\_OPEN, \newline IS\_DOOR\_OPEN, IS\_TEMP\_BELOW\_LOWERTHRESHOLD, IS\_TEMP\_ABOVE\_UPPERTHRESHOLD}, \tiny{OWNER\_DISTANCE (double)}\\ \hline
\end{tabular}
\caption{\small{An example smart home with three smart systems. Each sensor variable has type boolean unless otherwise indicated in parentheses.}}
\label{table:appliance_sensor}
\end{center}
\end{table*}


\subsection{Dependency policy}
\label{subsec:dep}
Let $k$ denote the controller for some dumb device and $c$ denote a command that it sends to the dumb device, based on the values of the sensors it relies on, jointly represented as $P$ (for properties). 
There are two types of policies of interest:
\begin{itemize}
\item \textit {policies based on complete specification}: A command is sent if and only if some conditions hold. This is represented as: $k ~\textbf{sends}~ c~\Leftrightarrow~P$.  

For example:
the garage door opens if and only if (the owner's car is approaching from outside, and is within a certain distance from the garage, and the owner is inside the car) or (the car is inside, and it is started, and the owner is inside the car).

\item \textit {policies based on partial specification}:  This is represented as: $k ~\textbf{sends}~ c \Rightarrow P$. This means that if the command is sent, then the conditions are true. It is possible that the command is not sent even when the conditions are true. 

For example: If the laundry machine runs it must be full, but the machine may not run as soon as it is full.
\end{itemize}
The generic structure of the above two types of policies are respectively:\\

\hspace{2cm}$(device\_controller) ~\textbf{sends}~(c) \Leftrightarrow$

\hspace{2cm}$\langle \mathit{sensor\_variable = state}\rangle^{+}$ \\

\hspace{2cm}$(device\_controller) ~\textbf{sends}~(c) \Rightarrow$

\hspace{2cm}$\langle \mathit{sensor\_variable = state}\rangle^{+}$ \\

\subsection{Control policy}
\label{subsec:control}

A dumb device may be controlled by more than one controller in the sense that one controller may be capable of sending some commands to it while another controller may send some other commands. At the same time, a single controller may control more than one dumb device. We use the terms ``legal" and ``illegal" to refer to dumb devices which may and may not be controlled by a controller respectively.
There are several possible wrong scenarios:
\begin{enumerate}
\item a controller sending an executable command to an ``illegal" dumb device. For example, if the garage door controller sends \texttt{open\_maindoor} to the main door. The main door can execute the command \texttt{open} and since it is a dumb device, it simply executes the command without verifying the sender. This is a security loophole which can be exploited to gain access to the house. 

\item a controller sending an executable but wrong command to a ``legal" dumb device. For example, if the garage door controller sends \texttt{open\_maindoor} to the garage door. The garage door is capable of executing \texttt{open}, so it will open up but this is not the correct behaviour of the controller.

\item a controller sending non-executable commands to ``illegal" dumb device. For example, if the garage door controller sends \texttt{open\_garagedoor} to the coffee maker and the coffee maker can only execute start and stop. In this case since the command is not executable by the receiving dumb device, it  will never get executed but nevertheless it is an indication of incorrect behaviour of the controller.   
\end{enumerate}

The control policy ensures that the above scenarios do not happen.
Every controller $k$ has a list of $n > 0$ commands it can send: $ C = \{c_1, c_2, ..., c_n\}$ which is a part of its specification. As mentioned in \S\ref{subsec:discovery}, $k$ maintains a list of $m \geq 0$ ``legal" dumb devices and for every dumb device $d_i$,  a set of $l_i > 0$ commands it may send to them. Let this list be represented as: \\ $D_{m}= \{(d_{1}, c_{11}), ..., (d_{1}, c_{1l_1}), (d_{2}, c_{21}), ..., (d_{2}, c_{2l_2}), ..., (d_{m}, c_{m1}), ..., (d_m, c_{ml_m})\}$. \\

%\noindent Every dumb device, $d_{i} \in D_{m}$ has a set of $n_i \geq 0$ actions it can execute: $A_{n_{i}}=\{a_{i1}, a_{i2}, ... a_{i{n_{i}}}\}$. 
\noindent The control policy states that:\\

\noindent $\forall~\mathit{\texttt{action\_dumbdevice}}$ sent by $k$, \\
$\exists~ c_i \in C~|~\texttt{action\_dumbdevice} = c_i$, \\
$ \exists~ j \in \{1, 2, ..., m\}~|~ \mathit{\texttt{dumbdevice} = d_j}$ and\\ $\texttt{action} \in \{c_{j1}, c_{j2}, ..., c_{jl_j}\}$


\subsection{New device policy}
\label{subsec:newitem}
In our smart home architecture, we classified the devices into three categories depending on the roles they play in the house. This classification allows us to narrow down the verification to the controllers only. 
Since the sensors only broadcast changes in the values of the parameters they measure and do not have the ability to control any other device and the dumb devices are only capable of executing commands sent by the controllers, adding a new sensor or dumb device to the house does not require additional verification. 
When a new controller is added to the house, it would require verification of the two policies defined earlier: the dependency and control policies of the new controller. 

\noindent Let \texttt{verify\_dependency\_policy} and \texttt{verify\_control\_policy} be methods for verifying the dependency and control policies respectively.

As long as all the existing controllers are already verified, only verifying the new controller ensures that the entire smart homes satisfies the dependency and control policies. Due to this, our methodology for specifying and verifying smart homes can scale to homes with a large number of devices.

\section{Static Analysis}
\label{sec:analysis}
Static analysis is a way to provide compile time guarantees that the code does not violate correctness and security specifications. Static techniques have certain advantages over dynamic techniques (like testing or dynamic policy enforcement) such as it does not rely on the quality of the test suite and does not require the code to be executed. In case of safety critical applications, static analysis is better than dynamic analysis because approaches like testing may not capture all possible corner cases and may fail to detect certain bugs thereby having severe consequences while dynamic enforcement of policies may add unnecessary overhead thereby making the system unreliable. We performed a preliminary experiment with the automatic laundry system and found that the execution times were much higher with dynamic policy enforcement and hence decided to develop a static analyzer for the smart home. Figure~\ref{fig:dynamic} shows the execution times for the laundry system. 

\begin{figure}[htpb]
\begin{center}
\includegraphics[scale=0.4, trim = 0 0 0 0]{results_update.png}
\caption{Execution times of the laundry machine controller with (PolicyEnforcedUpdate) and without(Update) dynamic policy enforcement.}
\label{fig:dynamic}
\end{center}
\end{figure}
We verify the following three properties using static analysis:
\begin{enumerate}[topsep=0pt,itemsep=-1ex,partopsep=1ex,parsep=1ex]
    \item A command is sent only within an \texttt{update()} method.
    \item A command is sent only after checking certain conditions inside an \texttt{if-statement}.
    \item A conditions checked in the \texttt{if-statement} are correct according to the policy file of the respective controller.
\end{enumerate}

\section{Implementation}
The implementation has two parts: a prototype of the smart home architecture described in section~\ref{sec:archi} and a static analysis tool that verifies whether the controllers in a smart system comply with their security policy specifications. The entire implementation is done in java. The smart home architecture implementation has approx. 1300 LOC\footnote{The code is here: https://gitlab.cs.washington.edu/cnandi/my-stuff} and the static analyser has approx. 300 LOC.  
For the static analysis tool, we built a checker on top of Google's error-prone framework~\cite{errorprone}. Error prone is a static analysis tool for Java and provides an API which can be used to develop different types of program analysis tools. As a part of the static analysis, we also developed a dataflow analysis for verifying the third item mentioned in section~\ref{sec:analysis}. The dataflow analyser parses the conditions inside the \texttt{if-statement} and its body, and compares them to the policies of the respective controller in the xml file and verifies if they are the correct conditions for the command being sent inside the body of the \texttt{if-statement}. 

\section{Results}

\section{Conclusions and future work}
\bibliographystyle{abbrv}
\bibliography{citations}
\end{document}
